\chapter{Conclusion}

Designing an intuitive user interface for a complex app will always be a
challenging task, but for an app to survive in the competitive mobile
environment the developers must do their best to ensure this is the case.
While no alternative forms of testing can completely replace the lessons learnt
by working closely with users and observing their behaviour, strategies that
cut down on the amount of expensive and time consuming user testing needed
will always be welcome.

In this project I have demonstrated one such possible strategy, and shown 
that adapting existing techniques to a more automated, crowdsourced technology
can provide valuable insights into the way that people really interact with
an application. This way of testing should be cheaper, and possibly quicker, 
to perform, enabling usability testing to be integrated into the development
workflow in ways it could not be before. While there are definitely improvements that could be made,
and ideas that can be built upon - exploring data collection and presentation
possibilities alone could form the basis of another entire project - the library
as it is can already provide valuable feedback.
It could be used to enhance the rapid iteration of an interface, backing up new design decisions
almost immediately with real data, or help a small hobbyist developer
track down confusing or buggy areas of their app that previously went
completely unnoticed. Indeed it has already given the users of one app
a better experience, and I look forward to implementing the remainder
of the findings as soon as possibly can.