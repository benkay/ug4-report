\chapter{Background}

There are two distinct types of UI testing that can be performed when building
an application. One is ensuring the interface performs as according to the
specifications and does not crash unexpectedly, and the other is testing that
the interface as implemented according to the specifications is usable and
intuitive for a user. While the former type of testing can be harder and more
time consuming than testing units of code, it is still relatively easy to
automate - in the Android platform alone, there exist a number of existing
tools which can be used to accomplish this, for example the Robotium
\cite{robotium} framework, and monkey and monkeyrunner \cite{monkeyrunner}
applications that ship with the Android developer tools. Usability testing on
the other hand is not so simple, precisely because it is testing how
a \emph{human} reacts to the application compared to what is expected by the
application's developer. As such, most techniques rely on a human participant
or manual inspection of the interface.

\section{User (or Hallway) Testing}

The most common method of usability testing performed by application developers
is done by recruiting five or six participants who are unfamiliar with the
application being tested and chosen to cover the intended target audience as
widely as possible. It usually takes place in a controlled setting designed to
mimic, say, an office or living room, and the participant given tasks that are
intended to cover the use cases of an application. As an example, if
a calendar/organiser application were being tested then the participant might
be asked to ``enter a reminder about dinner out next friday''. While completing
the task they are told to think aloud about their reasoning behind the actions
taken, and how they expect the application to behave, hence it is known as
``Think Aloud'' testing. This method is based on work by K. A. Ericsson and H.
A. Simon \cite{ericsson1980verbal}, and gives information about how the
interface matches the participant's thinking, and highlights areas that need
improvement. An important part of using this technique is to ensure the
participants are not involved with the project in any way, otherwise they will
often already know how to accomplish the tasks given, missing ambiguities and
false paths.

\subsection{Remote Testing}

In the event that test participants cannot be located near to the test
location, videoconferencing and screen recording/webcam software can be
utilised to perform testing remotely. A few companies, such as
\url{usertesting.com} take advantage of this to offload the work of finding and
testing participants from the developer.  The cost of these services (at the
time of writing, from \$49 per participant for a mobile application), while
definitely worthwhile at key points of the development may be prohibitively
costly for a developer with a small budget to make use of often.

\section{Heuristic Evaluation}

Since there are many recognised common usability principles that apply to
software interfaces, an interface can be inspected by an expert reviewer and
judged as to how well it conforms to these principles (the ``heuristics'').
This method of testing is potentially cheaper than testing with actual users,
since it only requires one expert to perform (although better results may be
gained with more evaluators).  Due to this is can also be undertaken with
shorter notice and with greater frequency (depending on the availability of the
expert) than user testing.

Despite these advantages, heuristic evaluation cannot displace user testing
since the two methods do not catch all the same issues. In contrast, the two
methods are complementary, and can be used at different points in the
development process for maximal performance \cite{tan2009web, archer2010web}.

\subsection{Automated Methods}

Tools such as USEFul \cite{dingli2011useful}
\url{http://usabilitygeek.com/useful-a-framework-to-automate-website-usability-evaluation-part-2/}.
Automatically compare websites to a database of usability guidelines. Optimised
for desktop websites. No such service currently available for native mobile
apps.

\subsection{A/B Testing}
