\chapter{Prior Work}

There is considerable prior work into both improving the usability of databases. However, not all of this work provides practical, usable tools.

\section{General Usability}

A number of the prior papers\cite{Date, Jagadish2007} assert that the usability of a database system is just as important as the performance or features. This assertion if made based on the fact that the cost of people is far more than the cost of machines - now more than ever. The papers define usability as raising the level of abstraction of the user interface. This is done in order to allow more people to be able to use the software or to help current users more easily use the software.

The \emph{Database Usability} paper\cite{Date} outlines some plans and ideas for improving database usability. For example, text editors, voice inputs and outputs, and pictorial interfaces. I don't believe that all of these interfaces would be effective or practical. However, the text editor interface is an interesting idea and is similar to my project. I also believe that the pictorial interface may be something that I can use and implement.

The second paper, \emph{Making Database Systems Usable}, notices that improving usability is the key to improving user engagement. However, this assertion is made after observing the improvements in search engine quality over the past few years. It goes on to mention how the problem of querying a database is far more complex when applied to databases compared to search engines. Many of the current systems require bespoke software to reach the same level of usability.

\section{Improving Existing Solutions}

One of the papers\cite{Stolte2010} claims that databases that are easy-to-use generate and improve story-telling, debate, and conversations. Additionally it provides an interesting concept. It introduces a more effective query language that can be compiled to SQL in order to be compatible with many different database systems by speaking their language directly. This solution is similar to part of my project that I will discuss later.

\section{Different Kinds of Visualisation}

One method of improving usability is providing different methods of visualisation. The rest of the papers\cite{Hu2008,Haritsa2010,Yang2010,Jin2010,Chan2009} are introducing different kinds of visualisation for databases.

\emph{QueryScope}\cite{Hu2008}, \emph{GBLENDER}\cite{Jin2010}, and \emph{Picasso}\cite{Haritsa2010} provide a method of visualising the queries that are passed to the database. This allows for similar query patterns to be identified and, if required, optimised. This is a method of improving usability that I did not want to pursue as it does not improve the usability for all users of database management systems. This method only improves the usability for those that enter queries directly such as database administrators.

Other projects\cite{Yang2010,Chan2009} improve the usability of these systems by allowing the data in the database to be viewed in different manners. This is the type of usability improvement that I am interested in working on as it improves the system for all users. One of the systems\cite{Yang2010} allows for the generation of various kinds of graphs and visual representations to be created from the data. The other solution, \emph{Vispedia}\cite{Chan2009}, allows for the querying and viewing of semi-structured date in a structured way. The example that is given in the paper is \emph{Wikipedia}. Using the system they have designed they are able to, for instance, ask for the population of Russia and it will look up the information even though the data is not strictly structured. This concept is very interesting however I will explore a different method in my project.
