\chapter{Introduction}

Databases are used at the core of nearly all enterprise software, most popular
websites and many common desktop applications and yet they are always hidden
behind layers of expensive, time consuming interface design and bespoke
software. Database systems are often avoided by even technical users, such as
developers, who will not use them directly instead preferring to use
technologies such as \acp{ORM} (Hibernate, ActiveRecord, etc.) or alternate,
more user-friendly, query languages than \ac{SQL} such as \ac{HQL}. If
technical users are avoiding databases then non-technical users, who may not be
aware of their widespread existence, have an even smaller chance of being able
to use them.

The databases query language of choice, \ac{SQL}, is an complex language that
is useful for lower level usage such as debugging a particular query or the
study of databases. Not helping this matter are the many different dialects
that each \ac{DBMS} requires that the user conforms to. This being one of the
reasons that \acp{ORM} are used so widely.

\ac{SQL} is not suitable for non-technical users to perform any common tasks.
There is a need for tools that help all users easily interact with databases.
While there are already tools that provide a front end to help more technical
users manipulate them at a lower level these tools require the user to have an
understanding of some core database concepts such as the difference between
a table and a view or when to create an index in order to maintain maximum
performance from the database.

This is not acceptable in the general case. There should be a way of allowing
any user who is familiar with the basic usage of a computer to be able to
accomplish a significant proportion of the tasks that the database can help
with. The aim of this project is to develop a system which is capable of this.
