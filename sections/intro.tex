\chapter{Introduction}

Usability and user experience are major concerns in modern application
development. With the proliferation of mobile devices and the emergence of competitive
app markets this has become an even bigger focus, as developers strive
create polished and intuitive experiences that explore the new user
interface (UI) paradigms suited to touch and smaller form factors.

Despite the simpler functionality that mobile apps tend to have compared to
their desktop counterparts, creating an interface that is obvious to a newcomer
to the application can still be a challenge. This is made worse by the relative
recency and pace of evolution of mobile devices meaning that UI conventions are
still in flux, and the limited screen real estate available on many devices.

The situation is not helped by the fact that testing the usability of an
application's interface is usually one of the most time consuming and resource
intensive parts of the testing process. Most other aspects of testing, such as
unit tests and integration tests can easily be, and are commonly, automated and
run without any interaction from a human, since they are testing the
correctness of the code in the application. Even coverage testing, which
stresses also the interface of an application, can run without intervention by
either following a predefined model of interaction, or randomly tapping areas
of the UI for a period of time looking for unexpected ways to cause a crash.

Since it has such an overhead to set up and perform, individual hobbyist
developers or even small companies with limited resources might shy away from
performing usability testing. Even if it is done, it might not be done as
regularly as might be optimal to catch design faults early in the process.

This project builds on principals established in user testing, but attempts
to make the process more accessible, both in time invested and cost. It 
provides a library that can be easily integrated into an Android application,
harnessing the power of crowdsourcing to quickly gather data about pain
points in the app's experience. We find that this produces useful and
easily interpretable data, showing potential to be integrated into the development
process. This would allow for rapid iteration of the interface, helping to
create a more usable Android app.
