\chapter{Introduction}

Usability and user experience are major concerns in modern application
development. With the proliferation of mobile devices and the explosion in the
app market this has become an even bigger focus for developers as they strive to
out compete others with polished and intuitive apps that explore new user
interface (UI) paradigms suited to touch and smaller form factors.

Despite the simpler functionality that mobile apps tend to have compared to
their desktop counterparts, creating an interface that is obvious to a newcomer
to the application can still be a challenge. This is made worse by the relative
recency and pace of evolution of mobile devices meaning that UI conventions are
still in flux, and the limited screen real estate available on many devices.

This situation is not helped by the fact that testing the usability of an application's interface is usually one of the most time
consuming and resource intensive parts of the testing process. Most other
aspects of testing, such as unit tests and integration tests can easily be, and are commonly, automated and run without any
interaction from a human, since they are testing the correctness of the code in
the application. Even coverage testing, which stresses the interface of the application, is run without intervention by
either following a predefined model of interaction, or randomly tapping areas of
the UI for a period of time looking for unexpected ways to crash the
application. On the Android platform alone, there exist a number of existing tools which can be used to
accomplish this, for example the Robotium \cite{robotium} framework and monkey
and monkeyrunner \cite{monkeyrunner} applications that ship with the Android
developer tools. Usability testing on the other hand is not so simple, precisely
because it is testing how a \emph{human} reacts to the application compared to
what is expected by the application's developer.

The most common way of performing usability testing done by application developers is by recruiting participants who are
unfamiliar with the application being tested and chosen to cover the intended
target audience as widely as possible. It usually takes place in a controlled
setting designed to mimic, say, an office or living room, and the participant
given tasks that are intended to cover the use cases of an application. 
As an example, if a calendar/organiser application were being tested then the participant might be
asked to ``enter a reminder about dinner out next friday". While completing the task they are
told to think aloud about their reasoning behind the actions taken, and how they expect the
application to behave. This ``Think Aloud" testing method is based on
work by K. A. Ericsson and H. A. Simon \cite{ericsson1980verbal}. It gives
information about how the interface matches the participant's thinking, and
highlights areas that need improvement.

Since it it has such an overhead to set up and perform, individual hobbyist
developers or even small companies with limited resources might shy away from
performing usability testing. Even if it is done, it might not be done as
regularly as might be optimal to catch design faults early in the process.

\todo{Briefly mention automated testing}

\todo{Briefly outline how it could be improved}