\chapter{Introduction}

Even though databases are used at the core of nearly all enterprise software, most popular websites and many common desktop applications they are always hidden behind layers of expensive, time comsuming interface design and bespoke software. Databases are avoided by even technical users such as developers who will often not use them directly instead preferring to use technologies such as \acfp{ORM} such as Hibernate, ActiveRecord or alternate, more user-friendly, query languages than \ac{SQL}.

\ac{SQL} is an complex language that is useful for lower level use such as debugging a particular query or the study of databases. Not helping this matter are the many different dialects that each \ac{DBMS} imposes that the user conforms to.

\ac{SQL} is not suitable for non-technical users to perform any common tasks. There is a need for tools that help all users easily interact with databases. While there are already tools that provide a front end to help more technical users manipulate them at a low level these tools require the user to understand the difference between a table and a view or when to create an index in order to maintain maximum performance from the database.

This is not acceptable in the general case. There should be a way of allowing any user who is familiar with the basic usage of a computer to be able to accomplish a good proportion of the tasks that the database can help with.

\section{Prior Work}

There is prior work into improving database usability\cite{database-use, Jagadish2007} but these only discuss the current problems with the database management systems and mention possibilities about how they could be improved without actually creating any of the systems. It does however offer up interesting ideas such as integrating database management systems with other graphics and word processing packages.

TODO: Find more prior work.