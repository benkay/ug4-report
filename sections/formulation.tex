\chapter{Formulation of the Problem}

The main obstacles to effective usability testing on the Android
platform are cost, expertise and the lack of effective automation.
If performing think aloud testing, participants must be found and
then travel to the site of the testing (or the developer travel to
them), and are usually paid for their time.  While there are services
that take care of much of the process on behalf of the developer,
the process tends to be costly, especially if it is done often.

Other types of usability testing such as heuristic evaluation must
be performed by experts to be effective.

Because of these obstacles, usability testing may not be performed
often or even at all during the development of a mobile application,
causing usability issues in the user interface to go undetected.

\todo{Keeping people engaged}

\section{Proposed Solution}

While analytics services can gather large amounts of data from the
interaction of users with an application, the lack of structure can
make it difficult to draw meaningful conclusions as information
about the intent of the user is lost when the data is gathered.
Without knowing the intent of users as they are using the application,
the data gathered cannot reveal where features of an application
are used or not used because their usability is poor, or whether
they are simply less popular.

Tasks which always have a specific structured workflow, such as a
checkout process, are easier to analyse with existing analytics
services, as user intent is clear. By giving the user more structured
tasks it should be possible to collect useful data from other parts
of a mobile application.

By combining some of the methodologies traditionally used in usability
testing with analytics, it should be possible to provide a structured
environment that can easily be given to people unfamiliar with an
application, and to easily gather data which will identify potential
problem areas with respect to usability.

The solution will take the form of a framework that can be easily
integrated into any mobile application, with as little modification
to the application's code as possible. The developer will define
tasks for the user to complete with defined starting points in the
app to make aggregation of data easier. Users can then attempt to
complete the tasks independently without there needing to be a
member of the development team present, as is usual with usability
testing, and the results will automatically be uploaded for the
developer to view at a later time. By aggregating the data from
many tests common problem areas should be able to be identified and
investigated further.
