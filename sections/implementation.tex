\chapter{Implementation}

To allow easy integration into any Android app, the library is implemented
as an Android Library Project \cite{android-library}.  It also needs to be easy
to remove/disable if the developer does not want the testing functionality
present for the released version.

\section{Integrating the Library}

To specify tasks, the developer creates two json files which are
included in the application's \verb+assets/+ directory. These files specify all the tasks that are available to perform, and what order they will be performed in.

To know when to begin and end tasks, inject the library's user interface in the app, and gather data about the participant's performance some integration with the application's code is needed.
To keep this to a minimum, all interaction with
the library in performed using the static \verb/Gamify/ class. This allows for
easy removal of the library as the developer can just find all references to
this class within the main application code and remove them. Since most Android
developers use \emph{Proguard} while compiling release versions, a rule could be
integrated to automatically strip the function calls out of production versions.

\begin{minted}{text}
  -assumenosideeffects class com.example.Gamify {
    <methods>;
  }
\end{minted}

The library can also be disabled by passing \verb|false| when initialising. The
initialisation is done by calling \verb|init| in either the \verb|Application|
or initial \verb|Activity|'s \verb|onCreate| method, passing the two
aforementioned json files.

\begin{minted}{java}
  Gamify.init(this /* Context */, true /* Enabled */,
      "challenges.json", "sequences.json");
\end{minted}

\subsection{Defining tasks}

The ``challenges.json'' and ``sequences.json'' files define the tasks and
challenges that will be used in testing, identified by ids. Each task has a title, description, and optional instruction that appears at the top of the screen during a task, and defaults to the title. The sequences file defines groups of tasks, and which order they will be run in. It also specifies whether to reset the user's position in the application hierarchy after a task, or continue from where they finished the previous task.

\subsection{Integrating Tasks Into the Application}

To start a sequence,the developer can simply call a single method, specifying the ID of the sequence
to be started.

\begin{minted}{java}
  Gamify.startSequence("sequenceid");
\end{minted}

Triggering points that will complete tasks are achieved with another method
call, for example if one task was completed by playing a song in the app, then
the corresponding code might look like:

\begin{minted}{java}
  public void playSong(Song song) {
    Gamify.completeChallenge("playsong");

    //... Code to play the song
  }
\end{minted}

If the ``playsong'' task is the current task, then it will be completed,
otherwise the method call will be ignored.

Beyond this, in each \verb|Activity|'s \verb|onResume| and \verb|onPause|
methods, the library must be notified.

\begin{minted}{java}
  @Override protected void onResume() {
     super.onResume();

     Gamify.onActivityResume(this);
  }

  @Override protected void onPause() {
     super.onPause();

     Gamify.onActivityPause(this);
  }
\end{minted}

This is so user interaction can be tracked, and the library's overlays can be
shown above the foreground activity. Since applications are usually designed so
that all activities inherit from a single base activity, this should not add
much extra overhead for the developer.

\section{Viewing the Data}

A companion application will run on a server and receive results from
participants. This can be a simple web app that receives data sent by the
Android app, validates it, and stores it in a database.  A separate web
application will retrieve the results from the database an display them to the
developer.
ty