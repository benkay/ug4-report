\chapter{Implementation}

\section{Choice of Language \& Technologies}

The first big implementation decision that was made was the choice of language
that was used to create the system. Before the beginning of term I had made
a proof-of-concept of the system using \emph{C}. I found that the pace of
development was too slow and difficult to make rapid progress. As such
I decided to go in a different direction.

In order to have a shallower learning curve I decided to use a language that
I was familiar with: \emph{Ruby}. I've used Ruby extensively before on many
personal and coursework projects. Besides the experience benefits withe the
language there are other fantastic things about it.

\begin{description}

\item[Community \& Existing Libraries] \hfill

There is an extensive preexisting Ruby community that is extremely helpful and
active. The advantage of this in my case is that there are already many
libraries for the boilerplate or lower level code that I will be using.
Importantly it already had bindings for \ac{FUSE} so writing the filesystem
component of the project could be started immediately.

\item[Testing Culture] \hfill

Due to the dynamic nature of the language such as the ability to redefine and
augment existing classes at runtime it is possible for the programmer to create
situations where code performs differently based on possible hidden factors
that have occurred before the execution. This is an extremely powerful and
useful tool but it must be handled in the correct way to eliminate the risk of
error. One of the ways that has evolved to cope with this is a strong testing
culture that is mandated by the community for almost all Ruby projects.

There are numerous libraries that make testing Ruby code quick and easy such as
\emph{RSpec}. Additionally, large \emph{Ruby} frameworks such as \emph{Ruby on
Rails} have been developed from the ground up to allow for simple and native
testing. This culture inspires an Agile attitude to refactoring and improving
code quality.

\item[Portability] \hfill

\emph{Ruby} runs on a number of different operating systems and platforms. This
factored into my choice as I did not want the available systems that the
project could run on to be limited by the language. In the end, it was
\ac{FUSE} (see section \ref{ssec:filesystem}) that was the limiting portability
factor.

\end{description}

\emph{Ruby} was a good choice of language for implementing both the filesystem
and query language. However, to build the application that allows the user to
query the database a GUI toolkit was needed. Unfortunately \emph{Ruby} is not
a suitable language for this task as it does not have any mature, powerful,
and platform-independent GUI toolkits.

In order to develop this stage I decided to use \emph{Java} and the
\emph{Swing} interface library. This decision was made for two main reasons.

\begin{description}
  \item[Portability] \hfill

    The language that was used for this application must be able to run on
    any platform that the filesystem was used. One of the main advantages
    of using \emph{Java} in any project is that the same code and
    sometimes even the same executable can be run on multiple platforms
    and systems. This is thanks to the ``Write once, run everywhere.''
    design consideration\footnote{In reality this is more likely to be
    ``Write once, debug everywhere.''}.

  \item[Familiarity] \hfill
\end{description}

\subsection{Filesystem}
\label{ssec:filesystem}

\ac{FUSE} is often used when creating a filesystem for a novelty purpose. It
allows the developer to write code that responds to certain callbacks after it
has been mounted on the existing filesystem at a certain directory. For
example, when a directory is requested \ac{FUSE} will ask the developer's code
what objects are present in that directory by supplying it the argument of
a path. \ac{FUSE} then queries each of these objects to find out if they are
a file or a folder and if there are any permissions that would not allow them
to be shown. These results are then displayed to the user in their file
explorer as if they were actually entities on disk.

This has a major advantage over regular filesystem development which is
extremely difficult and requires extensive knowledge of the kernel to complete.

\subsection{Query GUI}

As discussed above, it isn't simply possible for a query interface to be added
to the filesystem and so a separate but integrated application needed to be
made in order to satisfy this requirement. I elected to use Java and Swing to
complete this task as it is a mature and stable \ac{GUI} toolkit that is
supported on a wide range of operating systems. It is more important that this
application is more portable than the filesystem itself because it would be
feasibly possible to mount the filesystem above over \ac{NFS} and only have
this application on the clients computer to interact with it.

% TODO: More.

\section{Development}

\subsection{Filesystem}

The first stage of development was to create the filesystem on which all of the
other projects are based. As mentioned above, over summer I had created
a proof-of-concept system that allowed the user to view a read-only filesystem
representing a database. This was written in \emph{C} and used the native
\emph{FUSE} bindings. Not being satisfied with the development time and
complexities of the language and library I switched to using \emph{Ruby} and
the \emph{FuseFS} library.

The \emph{FuseFS} library works by ``mounting'' a class in a certain directory.
This class is required to implement certain methods (though due to the lack of
type-safety in \emph{Ruby} this can all fall apart at runtime). These methods
are then called when the user interacts with the filesystem whereupon the
return value from the called function is interpreted by the library and shown
in the filesystem. An example of the ``Hello World'' filesystem is shown in
figure~\ref{fig:fusefs}. This filesystem contains a single file called
\texttt{hello.txt} which when read has the contents \texttt{Hello, World!}.

\begin{figure}
\begin{minted}[linenos]{ruby}
class HelloDir
  def contents(path)
    ['hello.txt']
  end

  def file?(path)
    path == '/hello.txt'
  end

  def read_file(path)
    "Hello, World!\n"
  end

  def size(path)
    read_file(path).size
  end
end
\end{minted}
  \caption{A Sample \emph{FuseFS} filesystem.}
  \label{fig:fusefs}
\end{figure}

Using this library I created the filesystem that allowed the user to manipulate
databases. Each table was represented as a directory in the filesystem and
inside were files representing the records in those tables. Each of these were
formatted as \texttt{csv} files for ease of development and portability between
many different spreadsheet editors.

It was noticed that there was no way to select only certain columns to display
in these CSV files rather than just displaying all of them. A new directory was
added inside the table that allowed a user to move columns (represented again
as files) between two folders denoting the used and unused columns. Initially
it was planned that the user would be able to drag the file between the two
directories in order to change this. However, it was found that when using
\ac{FUSE} a file move only looks like a file deletion and a new file write with
no link between them. To overcome this the method which the user changes the
selected columns was changed to use the delete action. This is not ideal and
should be later changed to a more suitable action.

Similar to the \texttt{/proc} filesystem which my project gained much of its
inspiration I added files that could be read to find out information about the
database. For example, this includes the name, list of tables, path, and string
encoding. This allows users to easily find out information by only reading
a file instead of interrogating the database.

\subsubsection{Filesystem DSL Library}

The more I developed this system the slower it became to add new functionality
as the code became more complex. During the Christmas break I decided to write
a new library for creating these filesystem definitions. I decided to start out
with the ``dream library'' by writing down (in \emph{Ruby}) how \emph{I} would
like to build the filesystem. I have included the example I originally wrote in
figure~\ref{fig:skra}. This new \ac{DSL} library is called
\emph{Skra}\footnote{Skr\'{a} is Icelandic for ``file'' or ``directory''}.

The example below creates a filesystem that is mounted at \texttt{/mount/point}
and contains a directory called \texttt{hello} which has read and write
permissions. Inside this directory are two files: one named \texttt{path} which
contains its own path and the other is called \texttt{time} which when read
will return the current time.

Using this library allowed me to develop the remainder of the project far more
quickly and easily. The library itself was an interesting program to write due
to its implementation as a DSL.

\begin{figure}
\begin{minted}[linenos]{ruby}
Filesystem.new "/mount/point" do
  directory "hello", :perms => [:r, :w] do
    file "path" do |path|
      # File contains its own path.
      path
    end

    file "time" do |path|
      # Shows the current time.
      Time.now.to_s
    end
  end
end
\end{minted}
  \caption{An example \emph{Skra} layout.}
  \label{fig:skra}
\end{figure}

\subsection{Query Language}

You may have noticed above that I have not mentioned how we query the database.
This is due to the query problem that was mentioned earlier in
section~\ref{sec:queryproblem}. In order to query the database we needed an
intermediate language which could be easily generated by another application.

To create this I again used the expressiveness of \emph{Ruby} to create the
language. This would then use the \emph{Arel}
library\footnote{\url{https://github.com/rails/arel}} from \emph{Ruby on Rails}
in order to generate the \ac{SQL}. On top of this library was a small amount of
code that allowed that converted from the input files into commands to generate
the relational algebra that could then be turned into \ac{SQL}.

% TODO: More.

\subsection{Query GUI}

In order to generate this language a \ac{GUI} needed to be created that allowed
users to draw the graph representation of their query.

TODO: More.
