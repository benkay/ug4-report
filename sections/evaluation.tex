\chapter{Evaluation}

To measure the usefulness of this method of testing, a number of
aspects of the library were tested and contrasted to other ways of
gaining insights into the usability of an Android application.

\section{Integration}

\subsection{Creating Tasks}

Setting up this part of the system shares much in common with the
``think aloud'' method of testing.  Usually the developer will
devise tasks for the participant which mimic common use cases for
the part of the application being tested. Similarly, with the library
the developer can use these same tasks but formalise them in the
JSON files given as arguments in the setup of the library.

\subsection{Enabling the Library to Gather Data}

The effort required to finish the integration will depend on the
architecture of the application being tested, and the level of
detail the developer wants the library to record. In the ideal case
it can take as little as four \todo{Make sure this is always correct}
lines of code, plus one statement for each of the defined tasks at
the point where they should be completed. For some apps and apps
that make use of fragments it may take more work, but in few cases
should it be more than a few more lines of code.

For a developer who is familiar with the application being tested
and has experience with the testing library this should not amount
to more than a few hours work (and can be done incrementally as the
application is being developed). While this integration cost is not
present when using other ways of usability testing, it can be
compared to the cost of having to write unit and integration tests,
and in a similar manner to these it hopefully pays off in the long
run.

\subsection{Think of a title}

As a small and unscientific test of the ease of integration into a
new application, the library was integrated into an updated version
of the application used for development, reusing the same tasks
\ref{fig:updated-application}. In the interim between the version
frozen for development and the latest version, the architecture of
the code was refactored significantly, as has much of the user
interface. Additionally, whereas the old application used a separate
activity for each screen, the latest makes use of a single activity
and swaps fragments between screens.

\todo{How much effort was this?}

\begin{figure}
 \missingfigure{Updated application} \label{fig:updated-application}
\end{figure}

\section{Ease of Testing}

This is the area where this library hopes to improve on ``think
aloud'' testing.

Again, the convenience this method provides depends on the application
being tested. Since it makes use of crowdsourcing, the quality of
results relies on having a number of people available to use it.
This might be trivial with an application which, for example, has
an existing beta program within an organisation. I this scenario
the testing library can be integrated into the version distributed
to the beta users and run when on the first use of the application,
or when a new feature to be tested in implemented.

If a something like this does not exist then there are other solutions
to this problem. One route would be Amazon Mechanical Turk (MTurk),
a well known web service where users perform short tasks for a small
payment - usually in the region of a few tens of cents (USD).
Although this route involves money it still compares very favourably
with the amounts that are usually paid to testing participants, or
to services such as usertesting.com (\todo{Some of this info in the
Other Stuff section}) which charges \$49 per user.

\section{Quality of the results}

To evaluate the effectiveness of the data gathered from the testing
process, a number of participants were given the test application
and told to complete all the tasks. The application used was the
same as the one used for designing and developing the library and
the tasks used are shown in Appendix \ref{apdx:evaluation} along
with the result graphs generated by the web service.

Since the application required some setup on the computer side,
which was not being tested, a laptop with the software installed
was provided to the participant. The tests were supervised to observe
how the participant coped with using the test system, but were told
not to speak during the test or voice any opinion on the usability
of the app itself.



Design of the test, how it was performed, how many people, what the
tasks were.

\todo{Write all these up}

Things noticed so far:

\begin{itemize}
  \item addalbumtoplaylist has 100\% abandon rate! Graph shows what screen users expected to find
        the option before abandoning the task.
  \item playlibrary had high abandon rates - surprised me but people had issues finding the menu.
  \item search shows multiple paths, some with high backtrack rates - indicates that some areas of
        search UI are good, but others need work. In this case it looks like it is hard to find a 
        popular song from the artist page.
  \item most other tasks have good completion rates
  \item viewalbum showed people completing the task but not in the way implied by the task description.
        Badly written task? Showed UI was ok for finding an album but didn't test the part of the
        interface the developer wanted to.
\end{itemize}


\todo{Potential usability problems with the library e.g. ease of
writing effective tasks - found sometimes not read or understood
properly. Give the example of the "play album" task, where people
usually completed the task in a way that's not stressing the parts
of the UI the developer wanted. Quite obvious what happened in the
data, but did people just not read the task or was the task an
unrealistic scenario?}
